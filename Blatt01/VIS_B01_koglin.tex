\documentclass[10pt,a4paper, xcolor=dvipsnames]{article} % [Schriftgröße, Papierformat] {Art des Dokuments}


%---------------------------------------------------------------------------------------
% Die folgenden sind grundlegende Packete, die man meistens braucht:
% für deutsch (z.B ä,ö,...)
\usepackage[ansinew]{inputenc}
\usepackage{enumerate} % für einfache Aufzählungen
\usepackage{amsmath}  % für mathematische Formeln und Symbole
\usepackage{amsfonts} % auch für Mathe
\usepackage{amssymb}  % auch für Mathe
\usepackage{amsthm}   % auch für Mathe
\usepackage{a4wide} % macht die Seitenränder kleiner
\usepackage{graphicx}
\usepackage{float}
\usepackage{xcolor}
\usepackage{listings}
\usepackage{lastpage}

\definecolor{dkgreen}{rgb}{0,0.6,0}
\definecolor{gray}{rgb}{0.5,0.5,0.5}
\definecolor{mauve}{rgb}{0.58,0,0.82}
\definecolor{light-gray}{gray}{0.25}

\lstdefinestyle{java}{
  language=Java,
  aboveskip=3mm,
  belowskip=3mm,
  showstringspaces=false,
  columns=flexible,
  basicstyle={\footnotesize\ttfamily},
  numberstyle={\tiny},
  numbers=left,
  keywordstyle=\color{blue},
  commentstyle=\color{dkgreen},
  stringstyle=\color{mauve},
  breaklines=true,
  breakatwhitespace=true,
  tabsize=3
}

%Kopf- und Fu�zeile
\usepackage{fancyhdr}
\pagestyle{fancy}
\fancyhf{}

%Kopfzeile links bzw. innen
\fancyhead[L]{VIS, Blatt 1 -  Merlin Koglin}
%Kopfzeile rechts bzw. au�en
\fancyhead[R]{\today}
%Linie oben
\renewcommand{\headrulewidth}{0.5pt}

\rfoot{Page \thepage \hspace{1pt} of \pageref{LastPage}}

\newlength\tindent
\setlength{\tindent}{\parindent}
\setlength{\parindent}{0pt}
\renewcommand{\indent}{\hspace*{\tindent}}

\begin{document} % Hier fängt logischerweise das Document an ;)


\section*{VIS �bungblatt 1}

\subsection*{Backup}

\subsubsection*{\textcolor{purple}{Aufgabe 1 (Pflicht, 3 P.)}}
\textit{Nennen Sie drei typische Probleme, die f�r eine Datensicherung sprechen.}\\

Typische Probleme sind u.a. 
\begin{enumerate}
	\item Hardwaredefekte, also z.B. Ausfall einer Festplatte
	\item Diebstahl von Datentr�gern wie Festplatten
	\item H�here Gewalt, also z.B. Feuer, Erdbeben etc.
\end{enumerate}

\subsubsection*{\textcolor{gray}{Aufgabe 2 (Optional)}}
\textit{Welche Einflussfaktoren sind bei der Wahl von Backupstrategie und Backup-
medium zu ber�cksichtigen?}\\



\subsubsection*{\textcolor{gray}{Aufgabe 3 (Optional)}}
\textit{Auf welche Punkte sollte bei der Wahl des Backup-Aufbewahrungsortes
geachtet werden?}\\



\subsubsection*{\textcolor{purple}{Aufgabe 4 (Pflicht, 4 P.)}}
\textit{Was ist der Unterschied zwischen einem inkrementellen und einem differenziellen Backup? Welche Vor- und Nachteile haben diese Varianten?}\\

Das inkrementelle Backup sichert alle Daten, die nach dem letzten Backup (Voll oder inkrementell) hinzugekommen sind.
Ein Vorteil ist hierbei die im Vergleich zu dem differenziellen Backup bessere Speichereffizienz, da nur neue Daten gesichert werden.
Zum Wiederherstellen eines vorherigen Standes m�ssen allerdings das Voll-Backup plus alle bisherigen inkrementellen Backups nacheinander ausgef�hrt werden, 
dies bedeutet ggf. mehr Aufwand.
Ist zudem ein inkrementelles Backup vor dem aktuellen Backup defekt/gel�scht, kann nur der Stand bis vor diesem wiederhergestellt werden.

Beim differenziellen Backup werden alle Daten, die nach dem letzten Voll-Backup hinzugekommen sind gesichert.
Dies kann zu h�herem Speicherbedarf f�hren, vereinfacht aber die Wiederherstellung.
Zudem haben differenzielle Backup als Abh�ngigkeit einzig das letzte Voll-Backup.

\subsection*{RAID}

\subsubsection*{\textcolor{purple}{Aufgabe 5 (Pflicht, 1 P.)}}
\textit{Der Einsatz von RAID-Systemen erspart im Regelfall nicht ein Backup.
Warum?}

RAID-Systeme k�nnen die Datensicherung verst�rken und ggf. die Verf�gbarkeit verbessern,
allerdings sch�tzen sie nicht vor versehentlichem bzw. gewollten L�schen von Daten, Viren und m�glichen Softwarefehlern.

\subsubsection*{\textcolor{gray}{Aufgabe 6 (Optional)}}
\textit{Sie suchen nach einer fehlertoleranten Speicherl�sung. Der Ausfall einer
Festplatte soll toleriert werden. In die engere Auswahl kommen RAID 1 und RAID 5. Vergleichen
Sie RAID 1 und RAID 5 hinsichtlich Fehlertoleranz, Performance und Kosten.}


\subsubsection*{\textcolor{purple}{Aufgabe 7 (Pflicht, 3 P.)}}
\textit{Sie verf�gen �ber 4 Festplatten mit jeweils 500 GB Speicherplatz. Geben
Sie f�r jede der folgenden RAID-Varianten an, wieviel Speicherplatz Sie f�r Daten nutzen k�nnen:
a) RAID 0, b) RAID 1, c) RAID 4, d) RAID 5, e) RAID 6 und f) RAID 10.}

\begin{enumerate}
	\item RAID 0 \\
			Ein RAID 0-System verteilt die Daten auf den 4 Festplatten (ohne Redundanz). Hier sind also $4\cdot 500$GB = \textbf{2TB} Speicherplatz nutzbar.
	\item RAID 1 \\
			Werden die Daten im RAID 1 nur einmal gespiegelt, w�ren  $2\cdot 500$GB =\textbf{ 1TB} Speicherplatz nutzbar. (Bei h�herer Spiegelungszahl entsprechend weniger)
	\item RAID 4 \\
			F�r RAID 4 wird eine Festplatte f�r die Parit�tsinformationen gebraucht, also sind $3\cdot 500$GB = \textbf{1.5TB} Speicherplatz nutzbar.
	\item RAID 5\\
			�hnlich wie zu RAID 4, die Parit�tsinformationen sind hier zwar verteilt, aber genauso gro�, also sind effektiv $3\cdot 500$GB = \textbf{1.5TB} Speicherplatz nutzbar.
	\item RAID 6\\
			Durch die \textit{Kreuzparit�t} werden hier zwei Festplatten f�r die Parit�tsinformationen gebraucht, d.h. es sind $2\cdot 500$GB =\textbf{ 1TB} Speicherplatz nutzbar.
	\item RAID 10\\
			Bei der verteilte Spiegelung werden 1 Festplatte f�r die Spiegelung und 2 f�r die verteilte Spiegelung genutzt, daher ist nur
			\textbf{500GB} Speicherplatz nutzbar.
\end{enumerate}


\subsubsection*{Aufgabe 8}
\textit{Gegeben ist folgendes RAID-4-System, bestehend aus drei Datenplatten (HDD-1 bis HDD-3) und einer Parit�tsplatte (HDD-P):
\begin{itemize}
	\item Inhalt HDD-1 01101011
	\item Inhalt HDD-2 10010011
	\item Inhalt HDD-3 00011011
	\item Inhalt HDD-P 11100011
\end{itemize}}

\subsubsection*{\textcolor{gray}{Aufgabe 8a (Optional)}}
\textit{Ist das RAID-4-System in einem konsistenten Zustand?}\\

\subsubsection*{\textcolor{purple}{Aufgabe  8b  (Pflicht,  4  P.)}}
\textit{Der  Inhalt  von HDD-1 soll  mit 10010111 �berschrieben  werden.
Berechnen Sie die neue Parit�t mit und ohne Small-Write Algorithmus.}\\


Ohne Small-Write:
\[ \text{HDD-1}_{new} \oplus \text{HDD-2} \oplus \text{HDD-3} = \text{HDD-P}_{new}\]
\[ 10010111 \oplus 10010011 \oplus 00011011 = 00011111\]

Mit Small-Write:
\[ \text{HDD-1}_{new} \oplus \text{HDD-1}_{old}  \oplus \text{HDD-P}_{old}  = \text{HDD-P}_{new}\]
\[ 10010111 \oplus 01101011 \oplus 11100011 = 00011111\]

\subsubsection*{\textcolor{purple}{Aufgabe 8c (Pflicht 6 P.)}}
Das Codewort im Bild lautet: Plattendiversit�t.\\

Ich habe ein einfaches Programm in Java geschrieben, welches nicht sehr dynamisch ist und keine Fehlerbehandlung durchf�hrt,
aber das Problem l�st:
\lstset{style=java}
\begin{lstlisting}
public class Main {
    private final static String disksFolder = ".../VIS/�bung/Blatt01/raid4/";
    private final static String disk0 = "disk0";
    private final static String disk2 = "disk2";
    private final static String diskP = "disk3";
    private final static int stripSize = 4096;

    public static void main(String[] args) throws IOException {
        final Path disk0f = Paths.get(disksFolder + disk0);
        final Path disk2f = Paths.get(disksFolder + disk2);
        final Path diskPf = Paths.get(disksFolder + diskP);

        byte[] disk0Data = Files.readAllBytes(disk0f);
        byte[] disk2Data = Files.readAllBytes(disk2f);
        byte[] diskPData = Files.readAllBytes(diskPf);

        Path data = Paths.get(disksFolder + "data");
        Files.write(data, readRaidDisks(disk0Data, restoreDisk(disk0Data, disk2Data, diskPData), disk2Data));
    }

    private static byte[] restoreDisk(final byte[] disk0Data, final byte[] disk2Data, final byte[] diskPData) {
        
        byte[] disk1Data = new byte[diskPData.length];

        //restoring disk with xor composition
        int i = 0;
        for (byte diskPDataB : diskPData) {
            disk1Data[i] = (byte) ((diskPDataB ^ disk0Data[i]) ^ disk2Data[i]);
            i++;
        }
        return disk1Data;
    }

    private static byte[] readRaidDisks(final byte[] disk0Data, final byte[] disk1Data, final byte[] disk2Data) throws IOException {
        final ByteArrayOutputStream outputStream = new ByteArrayOutputStream();

        //concatenate data strips to get an image of cute kittens
        for(int i = 0; i < disk0Data.length / stripSize; i++) {
            int from = stripSize * i;
            int to = from + stripSize;
            outputStream.write(Arrays.copyOfRange(disk0Data, from, to));
            outputStream.write(Arrays.copyOfRange(disk1Data, from, to));
            outputStream.write(Arrays.copyOfRange(disk2Data, from, to));
        }
        return outputStream.toByteArray();
    }
}


\end{lstlisting}


\end{document}