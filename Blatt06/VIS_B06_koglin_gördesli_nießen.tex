\documentclass[10pt,a4paper, xcolor=dvinames]{article} % [Schriftgröße, Papierformat] {Art des Dokuments}


%---------------------------------------------------------------------------------------
% Die folgenden sind grundlegende Packete, die man meistens braucht:
% für deutsch (z.B ä,ö,...)
\usepackage[ansinew]{inputenc}
\usepackage{enumerate} % für einfache Aufzählungen
\usepackage{amsmath}  % für mathematische Formeln und Symbole
\usepackage{amsfonts} % auch für Mathe
\usepackage{amssymb}  % auch für Mathe
\usepackage{amsthm}   % auch für Mathe
\usepackage{a4wide} % macht die Seitenränder kleiner
\usepackage{graphicx}
\usepackage{float}
\usepackage{color}
\usepackage{soul} 
\usepackage{listings}
\usepackage{lastpage}
\usepackage[hidelinks]{hyperref}
\usepackage[german, ngerman]{babel}
\usepackage{tikz}
\usepackage{tabularx}
\usepackage{ctable}

\usepackage[numbers]{natbib}
\bibliographystyle{ieeetr}

\definecolor{dkgreen}{rgb}{0,0.6,0}
\definecolor{gray}{rgb}{0.5,0.5,0.5}
\definecolor{mauve}{rgb}{0.58,0,0.82}
\definecolor{light-gray}{gray}{0.25}

\lstdefinestyle{java}{
  language=Java,
  aboveskip=3mm,
  belowskip=3mm,
  showstringspaces=false,
  columns=flexible,
  basicstyle={\footnotesize\ttfamily},
  numberstyle={\tiny},
  numbers=left,
  keywordstyle=\color{blue},
  commentstyle=\color{dkgreen},
  stringstyle=\color{mauve},
  breaklines=true,
  breakatwhitespace=true,
  tabsize=3
}

\definecolor{lightgrey}{rgb}{0.90,0.90,0.90}
  \lstset{
    tabsize=2,
    escapeinside={(*@}{@*)},
    captionpos=t,
    framerule=0pt,
    backgroundcolor=\color{lightgrey},
    basicstyle=\small\ttfamily,
    keywordstyle=\small\bfseries,
    numbers=left,
    fontadjust,
    escapechar=�,
    breaklines=true
  }
  
 
  \newcommand\bh{\tikz[remember picture]
                    \node (begin highlight) {};
                 }
  \newcommand\eh{\tikz[remember picture]
                 \node (end highlight) {};
                 \tikz[remember picture, overlay] 
                 \draw[yellow,line width=10pt,opacity=0.3] (begin highlight) -- (end
                  highlight);
                 }


%Kopf- und Fu�zeile
\usepackage{fancyhdr}
\pagestyle{fancy}
\fancyhf{}

%Kopfzeile links bzw. innen
\fancyhead[L]{VIS, Blatt 6 -  Yunus G�rdesli, Kai Nie�en, Merlin Koglin}
%Kopfzeile rechts bzw. au�en
\fancyhead[R]{\today}
%Linie oben
\renewcommand{\headrulewidth}{0.5pt}

\rfoot{\thepage \hspace{1pt}}

\newlength\tindent
\setlength{\tindent}{\parindent}
\setlength{\parindent}{0pt}
\renewcommand{\indent}{\hspace*{\tindent}}

\begin{document} % Hier fängt logischerweise das Document an ;)

\subsection*{Aufgabe 1: Logische Uhren}
\begin{enumerate}[a)]


\item
\textit{In einem verteilten System treten die Ereignisse a bis l auf. Die kausalen
Abh�ngigkeiten sind durch das nachfolgende Happened-Before-Diagramm gegeben. Geben
Sie f�r alle im Happend-Before-Diagramm vorhandenen Ereignisse jeweils die Lamportzeit
und die Vektorzeit an.}
\\

\begin{center}

\begin{tabular}{|c|c|c|}
\hline
Ereignis  & Vektorzeit & Lamportzeit ($P_1$, $P_2$, $P_3$) \\
\specialrule{.1em}{.0em}{.0em}
a & 3 & (1, 0, 2)\\
\hline
b & 4 & (2, 0, 2)\\
\hline
c & 9 & (3, 5, 3)\\
\hline
d & 2 & (0, 1, 1)\\
\hline
e & 5 & (2, 2, 2)\\
\hline
f & 6 & (2, 3, 2)\\
\hline
g & 7 & (2, 4, 3)\\
\hline
h & 8 & (2, 5, 3)\\
\hline
i & 1 & (0, 0, 1)\\
\hline
j & 2 & (0, 0, 2)\\
\hline
k & 3 & (0, 0, 3)\\
\hline
l & 7 & (2, 3, 4)\\
\hline
\end{tabular}
\end{center}

\item
\textit{Geben Sie an, welche der folgenden Behauptungen f�r die Lamportzeit
zutrifft und welche der Behauptungen f�r die Vektorzeit zutrifft. Erl�utern Sie weiterhin
ihre Wahl.}\\
Eine Lamport Uhr erf�llt nur die schwachen Konsistenzbedingung f�r Uhren, d.h. f�r die Lamportzeit gilt Aussgae II), aber nicht I) und III). Man kann mithilfe der Lamportzeit also Ereignisse kausal nach dem Aufreten ordnen, aber keine Aussagen �ber Nebenl�ufigkeit und kausale Abh�ngikeiten treffen.

F�r die logische Uhr gelten alle drei Aussagen. Insbesondere erf�llt diese die starke Konsistenzbedingung f�r Uhren, d.h. I) und II) gelten beide.
Da durch Vergleichen der Zeiten kausale Abh�ngigkeiten erkannt werden k�nnen, kann auch eine kausale Unabh�ngikeit und damit Nebenl�ufigkeit erkannt werden. Somit gilt hier auch Aussage III).
\end{enumerate}

\subsection*{Aufgabe 3: Zeitsynchronisation}
\begin{enumerate} [a)]
	\item Sei $t$ der Zeitstempel des Servers und $T_{round}$ die Zeitspanne von Anfrage bis R�ckantwort. Dann ist:
\[T_{sync} = t + \frac{T_{round}}{2} = \text{10:14:05} + \frac{\text{10:14:06} - \text{10:14:00}}{2} =  \text{10:14:05} + \text{00:00:03} = \text{10:14:08}\]

	\item Die Berechnung nach Cristian beruht auf der Annahme, dass ein Paket f�r Hin- und R�ckweg die gleiche �bertragunszeit hat. In obiger Betrachtung wurde zudem angenommen, dass die Bearbeitung der Anfrage und Erstellung einer Antwort keine Zeit beansprucht.

\end{enumerate}


\subsection*{Aufgabe 4: Bully-Algorithmus}
\begin{enumerate}
	\item Der Bully-Algorithmus ist ein Auswahlalgorithmus der einen bestimmten Teilnehmer eindeutig bestimmt.
	In verteilten System kann hiermit z.B. beim Ausfall eines Koordinatorprozesses ein neuer Koordinator gew�hlt werden.
	Es wird hierbei angenommen, dass jeder Prozess die Id jedes anderen Prozesses im System kennt.
	Ein beliebiger Prozess, der den Ausfall des Koordinators registriert, z.B. durch ein Timeout, startet den Bully-Algorithmus indem er an alle Prozesse mit einer h�heren Prozess-Id eine Wahlnachricht sendet.
	Empf�ngt ein Prozess eine Wahlnachricht, sendet dieser wiederum eine Wahlnachricht an alle Prozesse mit h�herer Prozess-Id und eine Antwort an den Prozess, von dem die Wahlnachricht gesendet wurde.
	Falls ein Prozess die h�chste Prozess-Id besitzt oder keine Anwort von anderen Prozessen erh�lt, wird dieser der neue Koordinator.
	
	Bully-Algorithmus f�r vier aktive Knoten mit den IDs (3, 5, 9, 11):
	\begin{enumerate}
		\item $K_5$ beginnt und sendet Wahlnachricht an $K_9$, $K_{11}$.
		\item $K_9$, $K_{11}$ senden eine \textit{alive}-Antwort an $K_5$.
		\item $K_9$ sendet Wahlnaricht an $K_{11}$.
		\item $K_{11}$ besitzt die h�chste registrierte Prozess-Id und sendet seine eigene Wahl an die �brigen Prozesse.
	\end{enumerate}
	
\end{enumerate}

\end{document}